\usepackage{morefloats}
\definecolor{darkred}{rgb}{0.5,0,0}
\definecolor{darkgreen}{rgb}{0,0.5,0}
\definecolor{darkblue}{rgb}{0,0,0.5}
\hypersetup{ colorlinks,
linkcolor=darkblue,
filecolor=darkgreen,
urlcolor=darkred,
citecolor=darkblue }


\newcommand{\beq}{\begin{equation}}
\newcommand{\eeq}{\end{equation}}

\begin{document}

\author{
  Mohammadjavad~Vakili\altaffilmark{1},
  David~W.~Hogg\altaffilmark{1,2,3}
\altaffiltext{1}{Center for Cosmology and Particle Physics, New York University}
\altaffiltext{2}{Center for Data Science, New York University}
\altaffiltext{3}{Max-Planck-Institut f\"ur Astronomie}
}

\title{Probabilistic forward modeling of the PSF}

\begin{abstract}

\end{abstract}

\section{Introduction}


This paper is structured as follows. In section \ref{sec:method},
we give a brief overview of the centroiding methods used in our investigation.
In section \ref{sec:data}, we discuss the simulated data, and the CRLB on centroiding error.
In section \ref{sec:results}, we compare the results of our proposed method with 
those obtained from centroiding methods discussed in \ref{sec:method}, and 
with the CRLB derived in \ref{sec:data}. Finally, we discuss and conclude
in \ref{sec:discussion}.               

\section{Methods}\label{sec:method}

\end{document}
