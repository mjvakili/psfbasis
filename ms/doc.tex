\documentclass[12pt,preprint]{aastex}

% has to be before amssymb it seems
\usepackage{color,hyperref}
\definecolor{linkcolor}{rgb}{0,0,0.5}
\hypersetup{colorlinks=true,linkcolor=linkcolor,citecolor=linkcolor,
            filecolor=linkcolor,urlcolor=linkcolor}
\usepackage{url}
\usepackage{algorithmic,algorithm}
\usepackage{amssymb,amsmath}
\usepackage{natbib}	% For citep and citep
\usepackage{amsmath}	% for \iint
\usepackage{bbm}
\usepackage[breaklinks]{hyperref}	% for blackboard bold numbers
\usepackage{morefloats}
\usepackage{bm}
\newcommand{\beq}{\begin{equation}}
\newcommand{\eeq}{\end{equation}}

\begin{document}

\title{Forward modeling of the PSF}
\section{Introduction}

This paper is structured as follows. In section \ref{sec:model},
we discuss our proposed data-driven model for the PSF estimation.
In section \ref{sec:opt}, we discuss the optimization method
 we use in order to infer the model parameters, and also
 the training and validation criteria for 
truncating the optimization. In section \ref{sec:data},
 we demonstrate application of our model in esimating
 the PSF for the data from \emph{Deep Lens Survey}. 
In section \ref{sec:sys} we perform systematic tests,
 relavant to weak lensing cosmology, 
to asses the quality of the results obtained from our model. 
Finally, we discuss and conclude in \ref{sec:discussion}.               

\section{Model}\label{sec:model}

We want to use $N$ observed stars for modeling the PSF. 
Each star postage-stamp has $D$ pixels, 
and can be represented by an $D$-dimensional vector 
$\mathbf{y}_{i}$:

\beq
\mathbf{y}=
  \begin{bmatrix}
    y_{11}, ... , y_{1D} \\
    ., ... , . \\
    ., ... , . \\
    y_{N1}, ... , y_{ND} \\
  \end{bmatrix},
\eeq

Therefore, all observations can be compactly represented by 
$N\times D$ matrix $\mathbf{y}$, such that each row $i$ 
is the observed star $\mathbf{y}_{i}$. The central pixel of 
the postage stamp is chosen such that the estimated centroid
of the star $(x_i , y_i)$ is in the central pixel. 
Let us denote the vector connecting the the center of the 
postage-stamp of star $i$ to its centroid by $(\delta x_{i},
\delta y_{i})$, and the flux of star by $f_{i}$. 
  
A common procedure in the PSF estimation involves the follwoing steps:
$(i)$ estimation of the mean-PSF along each pixel $j\in\{1,...,D\}$
by iteritavely rejecting the bad pixels (eg. pixels hit by cosmic rays),
 $(ii)$ replacing the rejected pixel of each star by the estimated mean-PSF 
along that pixel, $(iii)$ normalizing the flux of each star,
 $(iv)$ applying of a sub-pixel shift
to the stars such that the centroid of each star lies on center 
of the postage-stamp of that star, $(v)$ subtracting the mean from each
column of $\mathbf{Y}$, and applying PCA to $\mathbf{Y}$.

We want to model the per-pixel intensity of stars $y_{ij}$ with
a linear combination of $Q$ components $a_{iq}$ where $q=\{1,...,Q\}$, and $Q\ll D$.
\beq
y_{ij} = \sum_{q=1}^{Q} f_{i}K_{i}\big(a_{iq}g_{qj}\big) + \text{noise},
\label{lin}
\eeq
where $f_{i}$ is the flux of star $i$, $K_{i} = K_{i}(\delta x_{i} , \delata y_{i})$
is a linear operator that shifts the center of the linear 
combination model $\sum_{q=1}^{Q}a_{iq}g_{qj}$ to the centroid 
of star $i$, $g_{qj}$ is the $j$-th component of the $q$-th 
basis function, and $a_{iq}$ is projection
of star $i$ onto the $q$-th basis function. 
Hereafter, we compactly represent the set
of $\{a_{iq}\}$ by a $N\times Q$ amplitude matrix $A$, 
and the set of $\{g_{qj}\}$
by a $Q\times D$ basis matrix.

 In the absence of subpixel shifts of the PSF model
with respect to the data, different flux values, and per-pixel 
uncertainties, one could rewrite the model (\ref{lin}) as a 
simple matrix factorization $Y = AG$. In this case, after 
subtracting the mean-PSF, one can apply a SVD decomposition 
to the data $Y$, and constructs $G$ from the $K$ eigenvectors 
of the covariance matrix that are associated with the highest 
eigenvalues of the covariance matrix. Then, once can use a simple 
least-square fit to determine the amplitude matrix $A$. 

The shift operation on an image is a linear transformation. In order to
apply a shift operation on PSF models, we use cubic spline interpolation.
Let us denote the PSF model for the i-th star by a $D$ dimensional column 
vector $m_{i}$. The PSF model $m_{i}$ is constructed such that it is centered
on the center of its postage-stamp. The operation that shifts the model to the
center of the data $(x_{i} , y_{i})$ using a cubic spline interpolation can
be done by multiplying a $D\times D$ matrix $P_{i}$ with  the model $m_{i}$.
The shift matrix $P_{i}$, whose entries are determined by sub-pixel shifts $(\delta x_{i}
, \delta y_{i})$, is extremely sparse because it only couples nearby pixels of 
the PSF model $m_{i}$.

Given the form of cubic spline shift operation, we can write the linear model 
(\ref{lin}) as follows
\begin{eqnarray}
y_{ij} = f_{i} \sum_{k=1}^{Q}\sum_{l=1}^{D} a_{ik}g_{kl}P^{(i)}_{lj} +
         f_{i}\sum_{l=1}^{D} X_{l}P^{(i)}_{lj} + \text{noise}.
\label{model}
\end{eqnarray}

We need to estimate the $N\times Q$ basis matrix $G$, the $Q \times D$ amplitude 
matrix $A$, the flux values $f_{i}$, and the $D$ dimensional mean-PSF model.
This requires minimizing the $\chi^{2}$ function.

\begin{eqnarray}
\chi^{2} &=& - \log \, p(\mathbf{Y}|A,G,X,\{f\}) \\
         &=& \text{const} + \sum_{i,j=1}^{N,D} \frac{\Big[y_{ij} - 
             f_{i} \sum_{k=1}^{Q}\sum_{l=1}^{D} a_{ik}g_{kl}P^{(i)}_{lj} +
             f_{i}\sum_{l=1}^{D} X_{l}P^{(i)}_{lj}\Big]^{2}}{\sigma_{ij}^{2}},
\label{chi}
\end{eqnarray}
where $p(\mathbf{Y}|A,G,X,\{f\})$ is the likelihood function.


\section{Optimization}\label{sec:opt}

Prior to optimizing the $\chi^{2}$ (\ref{chi}), we find the centroids of stars
using some centroiding algorithm CITESDSS. Using the centroid estimates,
 we apply the \emph{shifting} and \emph{adding} procedure described in section \ref{sec:model}
to initialize the basis matrix $G$, and the amplitude matrix $A$. Then we initialize 
the flux values by summing over the birghtness of pixels in postage-stamp of each star.   
 
\item {\bf Updating the mean-PSF model $X$}\quad

Having found initial estimates of the flux values $\{f_{i}\}$, 
we can fit a mean model to the data, in order to update the $D$-
dimensional mean-PSF. That is, we take the second term in the 
linear model (\ref{model}) and fit for $X_{l}$ in the follwing way:

\begin{eqnarray}
X &\leftarrow& \mathcal{P}^{-1}\mathcal{F}, \\
\mathcal{P} &\equiv& \sum_{i=1}^{N} \frac{P^{(i)}^{T}P^{(i)}}{\sigma_{i}^{2}}, \\
\mathcal{F} &\equiv& \sum_{i=1}^{N} \frac{P^{(i)}^{T}y_{i}}{f_{i}\sigma_{i}^{2}}.
\end{eqnarray}

\item {\bf Updating the amplitude matrix $A$}\quad

After updating the mean-PSF $X$, we subtract the second term of the
linear model (\ref{model}) from the left-hand-side. Now, we have
\begin{eqnarray}
\tilde{y}_{ij} &=& f_{i}\sum_{k=1}^{Q}a_{ik}\tilde{g}^{(i)}_{kj}, \label{tildey} \\
\tilde{y}_{ij} &\equiv& y_{ij} - f_{i}\sum_{l=1}^{D} X_{l}P^{(i)}_{lj}, \\
\tilde{g}^{(i)}_{kj} &=& \sum_{l=1}^{D} g_{kl}P^{(i)}_{lj}.
\end{eqnarray}

Now, given equation (\ref{tildey}), we can update the amplitude matrix $A$ as follows:
\begin{eqnarray}
A_{i} &\leftarrow& \Big[\frac{\tilde{G}^{(i)}\tilde{G}^{(i)}^{T}}{\sigma_{i}^{2}}\Big]^{-1}
                   \Big[\frac{\tilde{G}^{(i)}\tilde{y}_{i}}{f_{i}}\Big], \\
\tilde{G}^{(i)} &\equiv& GP^{(i)},
\end{eqnarray}
where $A_{i}$ is the $i$-th row of $A$.

\item {\bf Updating the basis matrix $G$}\quad

Given the updated amplitude matrix, for each patch $i$, we can update the 
matrix $\tilde{G}^{(i)} = GP^{(i)}$ in the following way:
\begin{eqnarray}
\tilde{G}^{(i)} &\leftarrow& [\mathcal{A}^{(i)}]^{-1}\mathcal{Y}, \\
\mathcal{A}^{(i)}_{jl} &\equiv& \frac{A_{ij}A_{il}}{\sigma_{i}^{2}}, \\
\mathcal{Y}^{(i)}_{jl} &\equiv& \frac{A_{ij}y_{il}}{\sigma_{i}^{2}}.
\end{eqnarray}

Now, we use the new $\tilde{G}^{(i)}$ matrices to update the basis matrix $G$:
\beq
G &\leftarrow& \sum_{i=1}^{N} \frac{[P^{(i)}]^{-1}\tilde{G}^{(i)}}{N}.
\eeq

\item {\bf Updating the flux values $\{f_{i}\}$}\quad

Given the updated mean-PSF $X$, amplitude matrix $A$, 
and basis matrix $G$, we update each flux value $f_{i}$ 
by dividing the data vector $y_{i}$ by the PSF-model
shifted by $(\delta x_{i} , \delta y_{i})$.  

\section{Application to Deep Lens Survey}\label{sec:data}
\section{Systematic tests}\label{sec:sys}
\section{Discussion}\label{sec:discussion} 

\end{document}
