\documentclass[12pt, preprint]{aastex}
\usepackage{graphicx}	% For figures
\usepackage{natbib}	% For citep and citep
\usepackage{amsmath}	% for \iint
\usepackage{bbm}
\usepackage[breaklinks]{hyperref}	% for blackboard bold numbers
\usepackage{hyperref}
\hypersetup{colorlinks}
\usepackage{color}
\definecolor{darkred}{rgb}{0.5,0,0}
\definecolor{darkgreen}{rgb}{0,0.5,0}
\definecolor{darkblue}{rgb}{0,0,0.5}
\hypersetup{ colorlinks,
linkcolor=darkblue,
filecolor=darkgreen,
urlcolor=darkred,
citecolor=darkblue }
\newcommand{\beq}{\begin{equation}}
\newcommand{\eeq}{\end{equation}}


\begin{document}

\author{
  Mohammadjavad~Vakili\altaffilmark{1},
  David~W.~Hogg\altaffilmark{1,2,3}
\altaffiltext{1}{Center for Cosmology and Particle Physics, New York University}
\altaffiltext{2}{Center for Data Science, New York University}
\altaffiltext{3}{Max-Planck-Institut f\"ur Astronomie}
}

\title{Probabilistic forward modeling of the PSF}

\begin{abstract}

\end{abstract}

\section{Introduction}


This paper is structured as follows. In section \ref{sec:model},
we discuss our proposed data-driven model for the PSF estimation.
In section \ref{sec:opt}, we discuss the optimization method we use in order to 
infer the model parameters, and also the training and validation criteria for 
truncating the optimization. In section \ref{sec:data}, we demonstrate application of 
our model in esimating the PSF for the data from \emph{Deep Lens Survey}. 
In section \ref{sec:sys} we perform systematic tests, relavant to weak lensing cosmology, 
to asses the quality of the results obtained from our model. 
Finally, we discuss and conclude in \ref{sec:discussion}.               

\section{Model}\label{sec:model}

We want to use $N$ observed stars for modeling the PSF. Each star postage-stamp has $M$ pixels, 
and can be represented by an $M$-dimensional vector $\mathbf{y}_{i}$:

\beq
\mathbf{y}_{i}=
  \begin{bmatrix}
    y_{i1} \\
    . \\
    . \\
    . \\
    y_{iM} \\
  \end{bmatrix},
\eeq

Therefore, all observations can be compactly represented by $N\times M$ matrix
$\mathbf{y}$, such that each row $i$ is the observed star $\mathbf{y}_{i}$.
The central pixel of the postage stamp is chosen such that the estimated centroid
of the star $(x_i , y_i)$ is in the central pixel. Let us denote the vector
connecting the the center of the postage-stamp of star $i$ to its centroid by $(\delta x_{i},
\delta y_{i})$, and the flux of star by $f_{i}$. An uncertainty $\sigma_{ij}$ is
associated with $j$-th pixel of the $i$-th star. If the only contribution to
per-pixel uncertainty is uncorrelated Gaussian noise, we will have $\sigma_{ij} = \sigma_{i}\delta_{ij}$.
However, one can use a more general form of uncertainty to account for the effect
of outlier pixels such as those hit by Cosmic rays, or overlapping stars.
   
A common procedure in the PSF estimation involves the follwoing steps:
$(i)$ estimattion of the mean-PSF along each pixel $j\in\{1,...,M\}$
by iteritavely rejecting the bad pixels, $(ii)$ replacing the rejected
pixel of each star by the estimated mean-PSF along that pixel, $(iii)$ 
normalizing the flux of each star, $(iv)$ applying of a sub-pixel shift
to the stars such that the centroid of each star lies on center 
of the postage-stamp of that star, $(v)$ subtracting the mean from each
column of $\mathbf{Y}$, and applying PCA to $\mathbf{Y}$.

We want to model the per-pixel intensity of stars $y_{ij}$ with
a linear combination of $Q$ components $a_{iq}$ where $q=\{1,...,Q\}$, and $Q\ll M$.
\beq
y_{ij} = \sum_{q=1}^{Q} f_{i}K_{i}\odot\big(a_{iq}w_{qj}\big) + \sigma_{ij},
\eeq
where $f_{i}$ is the flux of star $i$, $K_{i} = K_{i}(\delta x_{i} , \delata y_{i})$
is a linear operator that shifts the center of the linear combination
$\sum_{q=1}^{Q}a_{iq}w_{qj}$ to the centroid of star $i$, $w_{qj}$ is the 
$j$-th component of the $q$-th basis function, and $a_{iq}$ is projection
of star $i$ onto the $q$-th basis function. The shift
operation is done by interpolation, such as bilinear interpolation,
or cubic spline interpolation. 

Let us consider the case of bilinear interpolation. The model for each
star $i$ is $M$ dimensional vector $\mathbf{z}_{i}$ shifted by the operator $K_{i}$,
and multiplied by flux value $f_{i}$. One can rewrite the vector $\mathbf{z}_{i}$
as a $p\times p$ postage-stamp, where $p^{2}=M$. Applying a bilinear shift
operator $K(\delta x_{i} , \delta y_{j})$ results in another postage-stamp whose elements are given by
\begin{eqnarray}
\big(K_{i}\odot z_{i}\big)_{rs} &=& \delta x_{i} \delta y_{i} \, z_{i,r,s} \nonumber \\
                                &+& (1 - \delta x_{i}) \delta y_{i} \, z_{i,r+1,s} \nonumber \\ 
                                &+& \delta x_{i} (1-\delta y_{i}) \, z_{i,r,s+1} \nonumber \\ 
                                &+& (1 - \delta x_{i})(1-\delta y_{i}) \, z_{i,r+1,s+1} \, ,
\end{eqnarray}
where appropriate boundary conditions need to be used. 

In order to estimate the basis functions $\{w\}$, the amplitudes $\{a\}$,
and the flux values $\{f\}$, we need to minimize the chi-squared function
\begin{eqnarray}
\chi^{2} &=& - \log \, p(\mathbf{Y}|\{w,a,f\}) \\
         &=& \text{const} + \sum_{i,j=1}^{N,M} \frac{\Big[y_{ij} - 
             \sum_{q=1}^{Q} f_{i}K_{i}\odot\big(a_{iq}w_{qj}\big)\Big]^{2}}{\sigma_{ij}^{2}},
\end{eqnarray}
where $p(\mathbf{Y}|\{w,a,f\})$ is the likelihood function.


\section{Optimization}\label{sec:opt}



\section{Application to Deep Lens Survey}\label{sec:data}
\section{Systematic tests}\label{sec:sys}
\section{Discussion}\label{sec:discussion} 


\end{document}

